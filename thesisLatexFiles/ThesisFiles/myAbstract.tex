Renewable generation resources are one of the biggest trends emerging into the power systems world. The variability of these power sources brings challenges in terms of planning, reliability and feasibility factors. Classical formulations of Optimal Power Flow systems tries to generate power system at minimal costs, considering devices and transmission constraints.

This work evaluates the impact of different levels of green energy in terms of the two most common optimization models in power systems planning: the Unit Commitment and Economic Dispatch, under different time modelling and transmission assumptions. This thesis built a full user-friendly tool in AIMMS with 4 datasets based on the IEEE RTS-96 Test Cases with Wind and Solar profiles incorporated from the state of Texas, and it is going to be available to the academic community for power systems planning research purposes, with capacity to expand.

The analysis discusses that there, under high renewable resource penetration levels more factors has to be considered in the planning, such as ramping, under and over generation, storage behaviour, transmission line limits and start-up, shut-down profiles. It is also suggested that, under deep sub-hourly levels, the classic formulation of unit commitment is not efficient and, therefore is necessary to incorporate new optimization techniques.