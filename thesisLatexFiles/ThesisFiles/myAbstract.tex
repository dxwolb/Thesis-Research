\chapter*{Abstract}

Renewable power generation resources are one of the biggest trends emerging the power systems world. The inherent variability of these power sources brings challenges in terms of planning, reliability and feasibility factors. Classical formulations of Optimal Power Flow systems tries to generate power for a system at minimal cost, considering devices and transmission constraints.

This work evaluates the impact of different levels of renewable "green" energy in terms of the two most common optimization models in power systems planning: Unit Commitment and Economic Dispatch, under different time modelling and transmission assumptions.In this thesis a full user-friendly tool in AIMMS was built with 4 datasets based on the IEEE RTS-96 Test Cases with Wind and Solar profiles incorporated from the state of Texas. This tool will be available to the academic community for power systems planning research purposes, with capacity to expand.

Our analysis finds that, under high renewable resource penetration levels, more factors have to be considered in planning, such as ramping, under and over generation, storage behaviour, transmission line limits and start-up, shut-down profiles. It is also suggested that, under deep sub-hourly levels, the classic formulation of unit commitment is not efficient and, therefore it is necessary to incorporate new optimization techniques.