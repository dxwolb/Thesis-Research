%Sample of text file to include all the documents

\documentclass[12pt,LUDisStyle,twosided]{book}
\usepackage[hang,small,bf]{caption}
\usepackage{LUDisStyle}
\usepackage{graphicx}
\usepackage{subfigure}
\usepackage{amsmath}
\usepackage{amssymb}
\usepackage{amsthm}
\usepackage{verbatim}

\usepackage[square, comma, sort&compress]{natbib}
\renewcommand{\topfraction}{0.95}
\renewcommand{\textfraction}{0.05}
\renewcommand{\floatpagefraction}{0.85}
\addtolength{\belowcaptionskip}{10pt}
\newcommand{\mc}{\mathcal}


\begin{document}
\pagestyle{plain}
\pagenumbering{roman}
\dissertationtrue
\thispagestyle{empty}%

	\vskip0.5in%

	\begin{center}

		\LARGE Model fidelity and its impact on power grid resource planning under high renewable penetration

	\end{center}

	\vfill

	\begin{center}

		\rm by \\
		
                         \vfill
                 
		Daniel Xavier Wolbert \\

	\end{center}

	\vfill

	\begin{center}

		\rm A Thesis \\

		Presented to the Graduate Committee \\

		of Lehigh University \\

		in Candidacy for the Degree of \\

		Master of Science \\

		in \\

		Industrial and Systems Engineering

	\end{center}

	\vfill

	\begin{center}

		Lehigh University \\

		05/2016 \\

	\end{center}\vskip.5in
	
	\newpage

%put copyright info on second page
\vspace*{\fill}
\begin{center}
Copyright\\
Daniel Xavier Wolbert
\end{center}
\newpage
%Sample signature page

\thispagestyle{plain}

Approved and recommended for acceptance as a dissertation in partial fulfillment of the requirements for the degree of Doctor of Philosophy.
\\
\\
(Your Name) \\
(Title of Dissertation)

\vspace{.1in}

\begin{tabular}{l}
\\
\hline
\textbf{Date \ \ \ \ \ \ \ \ \ \ \ \ \ \ \ \ \ \ \ }
\end{tabular}

\begin{flushright}
\begin{tabular}{l}
\\
\hline
\textbf{(Committee Chair)}, Dissertation Director, Chair \\ 
\textbf{(Must Sign with Blue Ink)}
\end{tabular}

\vspace{.05in}

\end{flushright}

\begin{tabular}{l}
\\
\hline
\textbf{Accepted Date \ \ \ \ \ \ \ }
\end{tabular}


\begin{flushright}
\begin{tabular}{l}
Committee Members \ \ \ \ \ \ \ \ \ \ \ \ \ \ \ \ \ \ \ \ \ \ \ \ \ \
\\ 
\\
\\
\hline
\textbf{(Committee Member)}\\
\\ 
\\
\\
\hline
\textbf{(Committee Member)}\\
\\ 
\\
\\
\hline
\textbf{(Committee Member)}\\
\\
\\
\\
\hline
\textbf{(Committee Member)}\\
\end{tabular}
\end{flushright}
\newpage
\chapter*{Acknowledgements}

I would like to thank the Industrial and Systems Engineering department at Lehigh for providing me help and guidance. A special thanks to prof. Robert H. Storer as being my advisor and giving me the necessary support to achieve my objectives. Also, I would like to thank Dr. Dimitri Papageorgiou and his team to help me with all the data and necessary technical support.

I would like to thank the Brazilian government in the figure of CAPES to sponsor me and provide me with everything necessary to achieve my goals. Finally, I would like to thank my family, my future wife and my friends here and in Brazil for their continuous moral support in my achievements. I could not express in words how grateful I am.
\tableofcontents
\nopagebreak
\addcontentsline{toc}{chapter}{List of Tables}
\listoftables
\addcontentsline{toc}{chapter}{List of Figures}
\listoffigures
\newpage
\pagestyle{plain}
\pagenumbering{arabic}
\addcontentsline{toc}{chapter}{Abstract}
\include{myabstract}
\pagestyle{plain}

\chapter{Introduction}
\chapter{Literature Review}

\chapter{Development}
\section{Data}

\section{Models}

This thesis covers the Economic Dispatch and Unit Commitment models, developed exhaustively in the literature.

\subsection{Indices and Sets}

\begin{tabular}{ll}

$g \in \mc{G} $& Set of generators\\
$b \in \mc{B} $& Set of buses\\
$g \in \mc{G}^{NR} $& Subset of non-renewable generators\\
$g \in \mc{G}^{R} $& Subset of renewable generators \\
$gt \in \mc{GT} $& Set of generator types \\
$u \in \mc{U} $& Set of unit groups \\
$l \in \mc{L} $& Set of transmission lines \\
$rr \in \mc{RR} $& Set of reserve requirements \\
$rp \in \mc{RP} $& Set of reserve products \\
$t \in \mc{T} $& Set of time periods \\
$g \in \mc{G^{b}} $& Set of generators in each bus b \\

\end{tabular}

\subsection{Parameters}

\begin{tabular}{ll}

$D_{b,t} $& Load at bus $b$ in time $t$ \\
$C_{g} $& Generation cost for generator g (\$ / MW) $t$ \\
$S_{g} $& Start-up cost for generator g \\
$R^{up}_{g} $& Ramp up limit for generator g \\
$R^{down}_{g} $& Ramp down limit for generator g \\
$G^{max}_{g} $& Maximum generation capacity for generator g \\
$G^{min}_{g} $& Minimum generation capacity for generator g \\
$T^{min}_{l} $& Minimum transmission of transmission line l \\
$T^{max}_{l} $& Maximum transmission of transmission line l \\
$P^{R}_{g,t} $& Power generation of renewable generator g in time t\\


\end{tabular}


Note 1: The generation of renewable resources are not in the decision variables. The developed models only decides the generation for non-renewable resources. Therefore the generation is considered a deterministic parameter, not a variable in the model.

\subsection{Variables}

\begin{tabular}{ll}

$P^{NR}_{g,t} $& Power generation of non-renewable generator g in time t (MW)\\
$T_{i,j,t} $& Power transmitted from bus i to bus j in time t (MW)\\
$T^{loss}_{i,j,t} $& Power loss in transmission from bus i to bus j in time t (MW)\\
$S_{g,t} $& On/off status of generator g at time n\\
$S^{on}_{g,t} $& Start-up status of generator g at time n\\
$S^{off}_{g,t} $& Shut-down status of generator g at time n\\
$V^{-}_{b,t} $& Under generation slack variable at each bus b in time t\\
$V^{+}_{b,t} $& Over generation slack variable at each bus b in time t\\

\end{tabular}

\subsection{Economic Dispatch Models}

The economic dispatch model satisfies the load and transmission requirements at a minimum cost, following operational requirements such as generation, transmission and ramp limits. In this model, we assume that the commitment decisions has been already made.

\subsubsection{Simple Economic Dispatch Model}

\begin{subequations}\label{model:simple_ED}
\begin{alignat}{4}
min ~~& \sum_{t \in T}\sum_{g \in G^{NR}} P^{NR}_{g,t} C_{g} + \sum_{t \in T}\sum_{b \in B} V^{-}_{b,t} + \sum_{t \in T}\sum_{b \in B} V^{+}_{b,t} \label{eq:ObjectiveFunction} \\
s.t. ~~~& \sum_{t \in T} P^{NR}_{g,t} + \sum_{t \in T} P^{R}_{g,t} + \sum_{t \in T}V^{-}_{b,t} = \sum_{t \in T} D_{b,t}  + \sum_{t \in T}V^{+}_{b,t}  &~& \forall t \in T  \label{eq:loadBalanceConstraint} \\
& P^{NR}_{g,t} - P^{NR}_{g,t - 1} \leq R^{up}_{g} &~& \forall t \in T, g \in \mc{G}^{R}\label{eq:rampUpRateConstraint} \\
& P^{NR}_{g,t -1 } - P^{NR}_{g,t} \leq R^{down}_{g} &~& \forall t \in T, g \in \mc{G}^{R}\label{eq:rampDownRateConstraint} \\
& G^{min}_{g}\leq P^{NR}_{g,t} \leq G^{max}_{g} &~& \forall t \in T, g \in \mc{G}^{R}\label{eq:generationBounds}
\end{alignat} 
\end{subequations}

The constraint \ref{eq:loadBalanceConstraint} states the energy balance in every time. The constraints \ref{eq:rampDownRateConstraint} and \ref{eq:rampUpRateConstraint} states that every generator has to obey the ramp limits. The constraint \ref{eq:generationBounds} states the generation limits for each generator. The use of slack variables $V^{-}_{b,t}$ and $V^{+}_{b,t}$ is necessary to always have feasible solutions, and it is important for scenarios where there is a huge renewable penetrations, when there's an abrupt variation of generation and there is a over or under generation.   

\subsubsection{Economic Dispatch with Transmission Constraints}

In this case, the model has to consider transmission limits and losses, without transmission costs associated. The objective function \ref{eq:ObjectiveFunction} remains the same, and the constraints \ref{eq:rampUpRateConstraint}, \ref{eq:rampDownRateConstraint} and \ref{eq:generationBounds} are als used. The constraint \ref{eq:loadBalanceConstraint} is replaced by  

\begin{subequations}\label{model:edTransmissionConstraints}
\begin{alignat}{4}
&\sum_{g \in G^{b}} P^{NR}_{g,t} + \sum_{g \in G^{b}} P^{R}_{g,t} + \sum_{b^{in} \in B} T_{b^{in},b,t} + V^{-}_{b,t} = D_{b,t}  + V^{+}_{b,t} + \sum_{b^{out} \in B} T_{b,b^{out},t}  &~& \forall b \in B, t \in t \label{eq:transmissionBalanceConstraint} \\
& T^{min}_{l} \leq T_{b^{in},b^{out},t} \leq T^{max}_{l}  &~& \forall b \in B, t \in t \label{eq:transmissionLimits}
\end{alignat} 
\end{subequations}

Constraint \ref{eq:transmissionBalanceConstraint} guarantees that the energy balance is always satisfied for every bus: everything that is generated and received from other buses should be equal to what is loaded and sent throughout the line with the respective losses. If this balance is not satisfied then it is represented in the slack variables. (TODO: put lossed in the equation). The constraint \ref{eq:transmissionLimits} defines the transmission bounds for each line.

\subsubsection{Economic dispatch model with unit commitment constraints}
This optimization model is a mixed-integer linear program (MILP).
This model includes decision variables to capture ``on'' and ``off'' decisions for thermal generators in each time period. 
The objective function must be modifies to account for start up and shut down costs with thermal generators.

\section{Implementation}

Here write about how it was implemented, in AIMMS, put some running times in a table, some images, put a brief description about what is AIMMS and why it is used, and etc. If i Cite this one, will it apeaar? \cite{RefWorks:9}

\chapter{Analysis}

In the analysis make all the scenario runs, describe the data.

\chapter{Conclusion}

\appendix
%\include{appendix}

\nocite{RefWorks:11}
\bibliography{biblio}
\bibliographystyle{naturemag}
%\addcontentsline{toc}{chapter}{Vita}
%\chapter*{Vita}
\addcontentsline{toc}{chapter}{Vita}

Daniel Xavier Wolbert was born in 7 July of 1987 in Brazil, son of Cleber Wolbert and Mirian Xavier. He attended Universidade Federal de Minas Gerais in Brazil, graduating in December of 2010 in B.S. of Control and Automation Engineering. From 2010 to 2013, he acted as business and systems consultant at Accenture. He attended Lehigh University to obtain his M.S. in Industrial and Systems Engineering in May 2016. He was awarded as the "ISE Master's Student of The Year Award" in 2016.
\end{document}

